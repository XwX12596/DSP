\documentclass{article}
\usepackage{amsmath, amssymb, cite, algorithmic, url, braket}
\usepackage{graphicx}
\usepackage{pythonhighlight}
\usepackage[margin=1.5cm]{geometry}
\usepackage[title]{appendix}
\usepackage{listings}
\usepackage{booktabs}

\graphicspath{{../pic/}}
\lstset{
language=[ANSI]{C},
showtabs=true,
tab=,
tabsize=2,
basicstyle=\ttfamily\footnotesize,%\setstretch{.5},
stringstyle=\color{stringcolour},
showstringspaces=false,
alsoletter={1234567890},
otherkeywords={\%, \}, \{, \&, \|},
keywordstyle=\color{keywordcolour}\bfseries,
upquote=true,
morecomment=[s]{/*}{*/},
commentstyle=\color{commentcolour}\slshape,
literate=*%
{=}{{\literatecolour=}}{1}%
{-}{{\literatecolour-}}{1}%
{+}{{\literatecolour+}}{1}%
{*}{{\literatecolour*}}{1}%
{!}{{\literatecolour!}}{1}%
{[}{{\literatecolour[}}{1}%
{]}{{\literatecolour]}}{1}%
{<}{{\literatecolour<}}{1}%
{>}{{\literatecolour>}}{1}%
% {>>>}{\pythonprompt}{3}%
,%
frame=trbl,
rulecolor=\color{black!40},
backgroundcolor=\color{white},
breakindent=.5\textwidth,frame=single,breaklines=true
}

\begin{document}
\title{DSP Homework 08}
\author{Xu, Minhuan}
\maketitle
\tableofcontents
\begin{abstract}

\end{abstract}


\section{Comparison Between Two Sampling Methods}
\subsection{Understanding of Wan Sampling Method}
The Taylor Theorem is expressed as below.
$$
f(x) = \sum_{i = 0}^{n} ~ \frac{f^{(i)}(x_0)}{i!} ~ (x - x_0)^i + R_n(x)
$$

Applying to our signal $x(t)$, we can make $t_0 = nT, n = 2$ here and get
\begin{equation}
	x(t) = x(nT) + x'(nT)(x - nT) + \frac{x''(t_0)}{2}(x - t_0)^2 + R_2(t)
\end{equation}
What exactly we get after sampling is only the values of $x(nT)$, but It is easy to do derivation. Therefore, using simple circuit, we know the values of $x'(nT),\,x''(nT),\,x'''(nT),\,\cdots$

So, the reconstructed signal $\hat{x}(t)$ can be expressed as
\begin{equation}
	\hat{x}(t) = x(nT) + x'(nT)(x - nT) + R_1(t)
\end{equation}
but also as
\begin{equation}
	\hat{x}(t) = x(nT) + R_0(t)
\end{equation}
The difference of the above two lies on the error of the reconstructed signal. It time to do the quantitative analysis, we assume that
\begin{equation}
	\begin{aligned}
		&|\hat{x}(t) - x(t)| < \epsilon \\
		&|x^{(n)}(t)| \qquad < \eta_n
	\end{aligned}
\end{equation}
So, to ensure the accuracy, we need to make the sampling period $T$ satisfy the equations below.
\begin{align}
	T < \left\{ 
	\begin{array}{lr}
	 	\frac{2\epsilon}{\eta_1} & n = 1 \\
	 	2 \sqrt{\frac{2\epsilon}{\eta_2}} & n = 2
	\end{array} 
	\right.
\end{align}



\section{Conclusion}



\bibliographystyle{ieeetr}
\bibliography{../bib/database}

\begin{appendices}

\end{appendices}

\end{document}