\documentclass{article}
\usepackage{amsmath, amssymb, cite, algorithmic, url, braket}
\usepackage{graphicx}
\usepackage{pythonhighlight}
\usepackage[margin=1.5cm]{geometry}
\usepackage[title]{appendix}
\usepackage{listings}
\usepackage{booktabs}

\graphicspath{{../pic/}}
\lstset{
language=[ANSI]{C},
showtabs=true,
tab=,
tabsize=2,
basicstyle=\ttfamily\footnotesize,%\setstretch{.5},
stringstyle=\color{stringcolour},
showstringspaces=false,
alsoletter={1234567890},
otherkeywords={\%, \}, \{, \&, \|},
keywordstyle=\color{keywordcolour}\bfseries,
upquote=true,
morecomment=[s]{/*}{*/},
commentstyle=\color{commentcolour}\slshape,
literate=*%
{=}{{\literatecolour=}}{1}%
{-}{{\literatecolour-}}{1}%
{+}{{\literatecolour+}}{1}%
{*}{{\literatecolour*}}{1}%
{!}{{\literatecolour!}}{1}%
{[}{{\literatecolour[}}{1}%
{]}{{\literatecolour]}}{1}%
{<}{{\literatecolour<}}{1}%
{>}{{\literatecolour>}}{1}%
% {>>>}{\pythonprompt}{3}%
,%
frame=trbl,
rulecolor=\color{black!40},
backgroundcolor=\color{white},
breakindent=.5\textwidth,frame=single,breaklines=true
}

\begin{document}
\title{DSP Homework 07}
\author{Xu, Minhuan}
\maketitle
\tableofcontents
\begin{abstract}

\end{abstract}

\section{Pros and Cons of Shannon/Nyquist sampling method}
\subsection{Positive}
\subsubsection*{Ideal and easy to understand}
In Shannon/Nyquist sampling method, we only consider things that are ideal. Like $\delta(t)$, and the ideal LPF. It is mathematically simple and friendly to beginners.
\subsubsection*{Easy to calculate}
Using Fourier series and Fourier transform, we can easily find the distribution of energy on the frequency domain since the $\delta$ function has nice special properties.

\subsubsection*{No Error}
Through calculation, we know that if we make $f_s$ and $f_c$ have proper values, a perfectly the same frequency spectrum can be reconstructed. This is good because we don't need to worry about making mistake through communication.

\subsubsection*{Easy to Reconstruct}
Assuming that we have a sampled signal, all we should do is to find the bandwidth, and to let the sampled signal pass a LPF. Then, a perfectly reconstructed signal is complete.

\subsection{Negative}
\subsubsection*{Non-existent signal}
Though the sampling progress is theoretically easy to carry out, the $\delta$ function is not so easy to generate. Mathematically, it has infinite energy in \emph{one} period. According to the law of conservation of energy, we can't make a signal like the $\delta(t)$, see (\ref{eq:energyOfPulse}).
\begin{equation}
E = \sum_{n = -\infty}^{\infty} T \to \infty
\label{eq:energyOfPulse}
\end{equation}

\subsubsection*{Waste of bandwidth}
In practical usage, engineers cannot find a ideal LPF, especially which has infinite differential (has a sharp drop). All we can find is those which have smooth differential, and LPF like these will occupy more bandwidth of channels. Then, because

$$
f_c < f_s - W
$$

Therefore, the sampling frequency $f_s$ need to be larger which is not good.



\section{Conclusion}



\bibliographystyle{ieeetr}
\bibliography{../bib/database}

\begin{appendices}

\end{appendices}

\end{document}