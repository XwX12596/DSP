\documentclass{article}
\usepackage{amsmath, amssymb, cite, algorithmic, url, braket}
\usepackage{graphicx}
\usepackage{pythonhighlight}
\usepackage[margin=1.5cm]{geometry}
\usepackage[title]{appendix}
\usepackage{listings}
\usepackage{booktabs}

\graphicspath{{../pic/}}
\lstset{
language=[ANSI]{C},
showtabs=true,
tab=,
tabsize=2,
basicstyle=\ttfamily\footnotesize,%\setstretch{.5},
stringstyle=\color{stringcolour},
showstringspaces=false,
alsoletter={1234567890},
otherkeywords={\%, \}, \{, \&, \|},
keywordstyle=\color{keywordcolour}\bfseries,
upquote=true,
morecomment=[s]{/*}{*/},
commentstyle=\color{commentcolour}\slshape,
literate=*%
{=}{{\literatecolour=}}{1}%
{-}{{\literatecolour-}}{1}%
{+}{{\literatecolour+}}{1}%
{*}{{\literatecolour*}}{1}%
{!}{{\literatecolour!}}{1}%
{[}{{\literatecolour[}}{1}%
{]}{{\literatecolour]}}{1}%
{<}{{\literatecolour<}}{1}%
{>}{{\literatecolour>}}{1}%
% {>>>}{\pythonprompt}{3}%
,%
frame=trbl,
rulecolor=\color{black!40},
backgroundcolor=\color{white},
breakindent=.5\textwidth,frame=single,breaklines=true
}

\begin{document}
\title{DSP Homework 06}
\author{Xu, Minhuan}
\maketitle
\tableofcontents
\begin{abstract}
    
\end{abstract}

\section{Summary and Thoughts}

\section{My Sampling Method}

\subsection{Restatement}
Use the Fourier series to develop a sampling method and compare it with the Shannon/Nyquist sampling method through
examples.
\subsection{Improvement}
In class, we are considering the energy of the Fourier transform of the $s(t)$, we have
\begin{equation}
s(t) = \sum_{n = -\infty}^{\infty} c_n \, e^{j2 \pi n fT}
\end{equation}
and we have
\begin{equation*}
\begin{aligned}
E &= \int_{-\frac{T}{2}}^{\frac{T}{2}} \, s(t) \cdot s^*(t) \, \mathrm{d}t \\ 
&= \int_{-\frac{T}{2}}^{\frac{T}{2}} \, \sum_{n = -\infty}^{\infty} c_n \, e^{-j2 \pi n fT} \cdot \sum_{m = -\infty}^{\infty} c_m e^{j2 \pi m fT} \, \mathrm{d}t \\ 
&= \int_{-\frac{T}{2}}^{\frac{T}{2}} \, \sum_{n = -\infty}^{\infty} c_m \cdot c_n ~ e^{j2 \pi (n - m) fT}  \, \mathrm{d}t \\ 
&= \sum_{n = -\infty}^{\infty} c_m \cdot c_n ~ \int_{-\frac{T}{2}}^{\frac{T}{2}} ~ e^{j2 \pi (n - m) fT}  \, \mathrm{d}t \\ 
\end{aligned}
\end{equation*}
and the integral of $e^{j2 \pi (n - m) fT}$ usually be $0$ except that $n - m$, so
$$
E =\sum_{n = -\infty}^{\infty} c_n^2 \cdot T
$$

I will try to prove that if $E < \infty$, we must ensure that $n \to \infty, c_n \to 0$, which means 
\begin{equation}
E < \infty \Leftarrow  n \to \infty, c_n \to 0
\label{eq:energyInfty}
\end{equation}

if that
$$
n  \to \infty, c_n \to C \, (C \neq 0)
$$
we can always find a big number $N$ which makes $c_n^2 > 0 \, (n > N) $, and we can easily find that
$$
\sum_{n = N}^{\infty} c_n^2 > (\infty - N) \cdot c^2_{n(min)} \to \infty
$$

So, $E\to \infty$ when $n  \to \infty, c_n \to C \, (C \neq 0)$
Therefore, Equa.\ref{eq:energyInfty} is proved.
And we know that
\begin{equation}
    c_n = \frac{1}{T} \int_{-\frac{T}{2}}^{\frac{T}{2}} s(t) \, e^{-j2\pi n f_s t} \, \mathrm{d}t
\end{equation}

So, as I find out last week in my weekly report, ff we let the square wave last the length of $2\tau$, we now have
\begin{equation}
    \begin{aligned} s(t) & = \sum_{n = -\infty}^{\infty} \mathrm{rect} (\frac{t - nT}{\tau}) \\
             & = \sum_{n =-\infty}^{\infty}F_n e^{j2 \pi n f_s t}  \\
             & = \sum_{n = -\infty}^{\infty}\left[ \frac{1}{T} \int_{-\frac T2}^{\frac T2}
              \mathrm{rect}(\frac{t}{\tau}) e^{-j2 \pi n f_s t} \mathrm{d}t \right] e^{j2 \pi n f_s t} \\
             & = \sum_{n = -\infty}^{\infty}\left[ f_s \int_{-\tau}^{\tau} e^{-j2 \pi n f_s t} \mathrm{d}t \right] e^{j2 \pi n f_s t}  \\
             & = \sum_{n = -\infty}^{\infty}2f_s\, \mathrm{sinc} \,( 2f_s n\tau)\, e^{j2 \pi n f_s t} \\
             \label{eq:samplingSquare}
    \end{aligned}
\end{equation}

Here, 
\begin{align*}
\begin{array}{lrl}
s(t) &=& \sum_{n = -\infty}^{\infty} \mathrm{rect} (\frac{t - nT}{\tau}) \\ 
c_n  &=& 2f_s\, \mathrm{sinc} \,( 2f_s n\tau) < \infty
\end{array}
\end{align*}  

% According to the document on the Internet\cite{sinc_sum}, I know that
% \begin{equation}
% \sum_{n = -\infty}^{\infty} \, sinc(nT) = \frac{\pi}{T}
% \end{equation}

Therefore
\begin{equation*}
\begin{aligned}
E &= T \cdot \sum_{n = -\infty}^{\infty} c_n^2 \\ 
&= T \cdot \sum_{n = -\infty}^{\infty} sinc^2(2f_sn\tau)
\end{aligned}
\end{equation*}

In help of code, I know that
$$
\int_{-\infty}^{\infty} sinc^2(2f_sn\tau) = \frac{\pi}{2f_s\tau} < \infty
$$

Therefore
$$
\sum_{n = -\infty}^{\infty} sinc^2(2f_sn\tau) < \infty
$$
we can now say that the $s(t)$ mentioned in Equa.\ref{eq:samplingSquare} is energy-limited.

If we look back on the $\delta(t)$ sampling, we have the energy $E$ of $s(t)$:
\begin{equation}
E = \sum_{n = -\infty}^{\infty} T \to \infty
\end{equation}
which is not realizable.

\section{Conclusion}
    \subsubsection*{}
    \subsubsection*{}

% \bibliographystyle{ieeetr}
% \bibliography{../bib/database}

\begin{appendices}
    \section{Code Listing}
    \begin{python}
    import numpy as np
    from sympy import symbols, integrate, sinc

    x, a = symbols('x, a')

    print(integrate((sinc(a*x))**2,(x,-np.inf,np.inf)))
    \end{python}
\end{appendices}

\end{document}