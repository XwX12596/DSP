\documentclass{article}
\usepackage{amsmath, amssymb, cite, algorithmic, url, braket}
\usepackage{graphicx}
\usepackage{pythonhighlight}
\usepackage[margin=1.5cm]{geometry}
\usepackage[title]{appendix}
\usepackage{subfigure}
\usepackage{listings}
\usepackage{booktabs}

\graphicspath{{../pic/}}
\lstset{
language=[ANSI]{C},
showtabs=true,
tab=,
tabsize=2,
basicstyle=\ttfamily\footnotesize,%\setstretch{.5},
stringstyle=\color{stringcolour},
showstringspaces=false,
alsoletter={1234567890},
otherkeywords={\%, \}, \{, \&, \|},
keywordstyle=\color{keywordcolour}\bfseries,
upquote=true,
morecomment=[s]{/*}{*/},
commentstyle=\color{commentcolour}\slshape,
literate=*%
{=}{{\literatecolour=}}{1}%
{-}{{\literatecolour-}}{1}%
{+}{{\literatecolour+}}{1}%
{*}{{\literatecolour*}}{1}%
{!}{{\literatecolour!}}{1}%
{[}{{\literatecolour[}}{1}%
{]}{{\literatecolour]}}{1}%
{<}{{\literatecolour<}}{1}%
{>}{{\literatecolour>}}{1}%
% {>>>}{\pythonprompt}{3}%
,%
frame=trbl,
rulecolor=\color{black!40},
backgroundcolor=\color{white},
breakindent=.5\textwidth,frame=single,breaklines=true
}

\begin{document}
\title{DSP Homework}
\author{Xu, Minhuan}
\maketitle
\tableofcontents
\begin{abstract}

\end{abstract}

\section{Videos}

\section{FIR And IIR}
\subsection{(a)}
\subsubsection{When $Q = 0$}
When $Q = 0$, the system can be represented as 
$$
y(n) = a_0x(n) + \cdots + a_Px(n - P)
$$
Then, the $h(n)$ can be written as
$$
h(n) =  \sum_i^P a_i \times \delta(n - i)
$$
So, $h(n)$ has finite duration, the system is FIR.
\subsubsection{When $Q > 0$}
When $Q > 0$, the system can be represented as 
\begin{equation}
y(n) + b_1 y(n - 1) + \cdots + b_Q y(n - Q) = a_0 x(n) + a_1 x(n - 1) + \cdots + a_P x(n - P)
\label{eq:iir}
\end{equation}
We can know that the $y(n), y(n - 1), y(n - 2), \cdots $ is 
\begin{gather}
y(n) = a_0 x(n) + a_1 x(n - 1) + \cdots + a_P x(n - P) - b_1 y(n - 1) - \cdots - b_Q y(n - Q) \label{eq:yn} \\ 
y(n - 1) = a_0 x(n - 1) + a_1 x(n - 2) + \cdots + a_P x(n - P - 1) - b_1 y(n - 2) - \cdots - b_Q y(n - Q - 1) \label{eq:yn-1} \\ 
y(n - 2) = a_0 x(n - 2) + a_1 x(n - 3) + \cdots + a_P x(n - P - 2) - b_1 y(n - 3) - \cdots - b_Q y(n - Q - 2) \label{eq:yn-2} \\ 
y(n - 3) = a_0 x(n - 3) + a_1 x(n - 4) + \cdots + a_P x(n - P - 3) - b_1 y(n - 4) - \cdots - b_Q y(n - Q - 3) \notag \\ 
\cdots \notag
\end{gather}

Equation (\ref{eq:yn}) contains all the terms in (\ref{eq:yn-1}), and (\ref{eq:yn-1}) contains all the terms in (\ref{eq:yn-2}), all the same below, so (\ref{eq:yn}) seems to have more terms than (\ref{eq:yn-1}). However, actually there are always the same number of terms in each equation, so (\ref{eq:yn}) does contain all the terms in (\ref{eq:yn-1}), but also it has the same number of terms compared with the equations below. Only possibility is that $y(n)$ and $y(n - 1)$ and so on have infinite number of terms. In one word, the system which (\ref{eq:iir}) reveals is IIR.
\subsection{(b)}
\subsubsection{Digital Filter Stability}
When a digital filter has the feature that when input is bounded, the output must be bounded, this digital filter is stable.
\subsubsection{FIR Stability}
FIR can be expressed as below
$$
y(n) = a_0x(n) + \cdots + a_Px(n - P)
$$
If $x(n)$ is bounded, we can always find a maximum of $|x(n)|$ which we call it $max\{ |x| \}$. And we can find the maximum of $|a_i|$ which we call it $A$ as well. So

\begin{align*}
y(n)  &= a_0x(n) + \cdots + a_Px(n - P) \\ 
&< |a_0||x(n)| + \cdots + |a_P||x(n - P)| \\ 
&< A\times max\{ x \} \times P \\ 
&< \infty
\end{align*}

\subsubsection{IIR Stability}
We have
\begin{align*}
	|y(n)| &= |x(n) * h(n)| \\ 
	& \leq \sum_{m} |x(m) \times h(m - n)| \\ 
	& \leq \sum_{m} |x(m)| \times |h(n - m)| \\ 
	& \leq \sum_{m} max\{ x \} \times h(n - m) \\ 
	& = max\{ x \}  \sum_{m} h(m)
\end{align*}
and if we need $y(n) < \infty$, we should have
\begin{equation}
\sum_{- \infty}^{\infty} h(n) < \infty
\label{eq:limit}
\end{equation}

Look at the $\tilde{h}(z)$ we can get from (\ref{eq:yn}):
\begin{equation}
	\tilde{h}(z) = \frac{\sum_{p = 0}^{P}a_p z^{-p}}{\sum_{q = 0}^{Q} b_q z^{-q}}
	\label{eq:hz}
\end{equation}

According to mathematics, we can rewrite $h(z)$ as below

\begin{equation}
\tilde{h}(z) = z^{\lambda - 1} \left[ \frac{z}{z - \alpha_0} + \frac{z}{z - \alpha_1 } + \cdots + \frac{z}{z - \alpha_N } \right]
\label{eq:hzz}
\end{equation}

First, we know in frequency domain, the $\frac{z}{z - \alpha} = \frac{1}{1 - \alpha/z} = \sum_{- \infty}^{\infty} az^{-1}$ means $\alpha^n$ in time domain.

Second, see (\ref{eq:hzz}) again, and in order to satisfy the requirement in (\ref{eq:limit}), we need to make all poles in (\ref{eq:hzz}) (which are $\alpha$ here) less than $1$, so that all $\alpha^n$ will converge, at last $h(z)$ will also converge.

\subsubsection{An Example in Class}
When $Q = 2, P = 0$, we have

\begin{equation}
y(n) + b_1y(n - 1) + b_2y(n - 2) = a_0 x(n)
\label{eq:yxExample}
\end{equation}
In this condition, $\tilde{h}(z)$ is like 
\begin{equation}
\tilde{h}(z) = \frac{a_0}{1 + b_1 z^{-1} + b_2 z^{-2}} = \frac{a_0 z^2}{z^2 + b_1 z + b_2}
\label{eq:hzExample}
\end{equation}
Here, the expression of $H(z)$ is as below
\begin{equation}
H(z) = z^2 + b_1 z + b_2
\label{eq:Hz}
\end{equation}
Roots of this equation is $z = -b_1 \pm \sqrt{b_1^2 - 4 b_2}$.

First, if (\ref{eq:Hz}) has real root(s), to make the root(s) be in (-1, 1), the sufficient and necessary conditions is as below
\begin{align*}
	\begin{cases}
		H(-1) = 1 - b_1 + b_2 > 0 \\ 
		H(1) = 1 + b_1 + b_2 > 0 \\ 
		- \frac{b_1}{2} \in (-1, 1)
	\end{cases}
\end{align*}

It is easy using linear programming to prove that all the conditions above can be derived from $|b_1| + |b_2| < 1$.

Second, if (\ref{eq:Hz}) has complex root(s), to make the root(s) be in (-1, 1), the sufficient and necessary conditions is as below

\begin{align*}
		|\alpha_1| = |\alpha_2| &= |\frac{-b_1 \pm j\sqrt{4b_2 - b_1^2}}{2}| \\ 
		&= \sqrt{b_2} \in (-1, 1)
\end{align*}
It is easy to find that $\sqrt{b_2} \in (-1, 1)$ according to $|b_1| + |b_2| < 1$. 

In conclusion, we can find that if $|b_1| + |b_2| < 1$ is true, the roots of $H(z)$ is all within the unit circle which means the system in (\ref{eq:hz}) is stable. So, $|b_1| + |b_2| < 1$ is a sufficient condition for the IIR system to be stable when $Q = 2, P = 0$.

\section{Why Some Voices Sound nicer}




\bibliographystyle{ieeetr}
\bibliography{../bib/database}

\begin{appendices}
\section{Code Listing}
\end{appendices}

\end{document}