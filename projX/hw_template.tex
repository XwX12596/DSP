\documentclass{article}
\usepackage{amsmath, amssymb, cite, algorithmic, url, braket}
\usepackage{graphicx}
\usepackage{pythonhighlight}
\usepackage[margin=1.5cm]{geometry}
\usepackage[title]{appendix}
\usepackage{subfigure}
\usepackage{listings}
\usepackage{booktabs}

\graphicspath{{../pic/}}
\lstset{
language=[ANSI]{C},
showtabs=true,
tab=,
tabsize=2,
basicstyle=\ttfamily\footnotesize,%\setstretch{.5},
stringstyle=\color{stringcolour},
showstringspaces=false,
alsoletter={1234567890},
otherkeywords={\%, \}, \{, \&, \|},
keywordstyle=\color{keywordcolour}\bfseries,
upquote=true,
morecomment=[s]{/*}{*/},
commentstyle=\color{commentcolour}\slshape,
literate=*%
{=}{{\literatecolour=}}{1}%
{-}{{\literatecolour-}}{1}%
{+}{{\literatecolour+}}{1}%
{*}{{\literatecolour*}}{1}%
{!}{{\literatecolour!}}{1}%
{[}{{\literatecolour[}}{1}%
{]}{{\literatecolour]}}{1}%
{<}{{\literatecolour<}}{1}%
{>}{{\literatecolour>}}{1}%
% {>>>}{\pythonprompt}{3}%
,%
frame=trbl,
rulecolor=\color{black!40},
backgroundcolor=\color{white},
breakindent=.5\textwidth,frame=single,breaklines=true
}

\begin{document}
\title{DSP Homework 10}
\author{Xu, Minhuan}
\maketitle
\tableofcontents
\begin{abstract}

\end{abstract}

\section{Videos}




\section{DSP Hardware Platforms}






\section{Relationship Between FT and DTFT}
We can first put the expression of perfect reconstruction here
\begin{equation}
x_s(t) = \sum_{n} \; x(nT) \, \delta(t - nT) = x_a(t) * s(t) =  x_a(t) * \sum_{n} \left[ \delta(t- nT) \right]
\label{eq:sampling}
\end{equation}
We put (\ref{eq:sampling}) into FT
\begin{equation}
\begin{aligned}
	\widetilde{x}_s(f) &= \int_{- \infty}^{\infty} \sum_{n} \; \left[  x(nT) \, \delta(t - nT) \right] \; e^{-j 2\pi ft}\; \mathrm{d} t \\
	&= \sum_{n} \; x(nT)  \, e^{-j 2\pi f nT} \\ 
	&=\sum_{n} \; x(n)  \, e^{-j 2\pi n f'}
\end{aligned}
\label{eq:DTFT_0}
\end{equation}
The value of $T$ (also $\frac{1}{f_s}$) doesn't matter, so we can make 
$$
f' = \frac{f}{f_s} 
$$
The (\ref{eq:DTFT_0}) can be changed into
\begin{equation*}
\widetilde{x}_s(f) = \sum_{n} \; x(n)  \, e^{-j 2\pi n f'}
% \label{eq:DTFT}
\end{equation*}
Therefore
\begin{equation*}
\begin{aligned}
\widetilde{x}_s(f) &= \mathcal{F} \left[ x_s(t) \right] \\ 
&= \mathcal{F} \left[ x_a(t) * s(t) \right] \\ 
&= \widetilde{x}_a(f) \times \widetilde{s}(f) \\ 
&= \widetilde{x}_a(f) \times \left[ f_s \sum_n \delta(f - nf_s) \right]
\end{aligned}
\end{equation*}

Therefore, the relationship can be described as
\begin{equation}
	\widetilde{x}(f) = \widetilde{x}_s(f) = \widetilde{x}_a(f) \times \left[ f_s \sum_n \delta(f - nf_s) \right]
\label{eq:relationship}
\end{equation}

\section{From Analog Signal to DFT}
Assuming there's an analog signal $x_a(t)$, computer cannot deal with analog values, so we can sample it use Shannon/Nyquist sampling method. 
\begin{equation}
x(n) = x_s(t) =  x_a(t) * \sum_{n} \left[ \delta(t- nT) \right] = \sum_{n} \; x(nT) \, \delta(t - nT)
\label{eq:Sampling_Mod}
\end{equation}
Then, we should analyze $x(n)$ in frequency domain, calculate the FT of $x(n)$ as below. Also, because computer cannot deal with analog values, so we should assume $n \in [0, N)$ to make $x_a(t)$ time-limited.
\begin{equation}
\begin{aligned}
	\widetilde{x}_s(f) &= \int_{- \infty}^{\infty} \sum_{n = 0}^{N - 1} \; \left[  x(nT) \, \delta(t - nT) \right] \; e^{-j 2\pi ft}\; \mathrm{d} t \\
	&= \int_{- \infty}^{\infty} \sum_{n = 0}^{N - 1} \; \left[  x(nT)  \, e^{-j 2\pi f nT} \, \delta(t - nT) \right] \; \mathrm{d} t \\
	&= \sum_{n = 0}^{N - 1} \; \left[  x(nT)  \, e^{-j 2\pi f nT} \, \int_{- \infty}^{\infty} \, \delta(t - nT) \, \mathrm{d} t  \right]\\
	&= \sum_{n = 0}^{N - 1} \; x(nT)  \, e^{-j 2\pi nT \cdot f} \\ 
	&= \sum_{n = 0}^{N - 1} \; x(n)  \, e^{-j 2\pi n \cdot f}
	% &=\sum_{n} \; x(n)  \, e^{-j 2\pi n f}
\end{aligned}
\label{eq:DTFT_deriv}
\end{equation}
Because now $\widetilde{x}_s(f)$ is discrete and we don't care about the sampling period $T$, so we make $f = f \times T$ in (\ref{eq:DTFT_deriv}).

However, $\widetilde{x}_s(f)$ is still not discrete. So, as what we did to $x_a(t)$, we should sample $\widetilde{x}_s(f)$ in frequency domain again. Rewrite the summation of (\ref{eq:DTFT_deriv}) as below:
$$
\widetilde{x}_s(f) = x(0) + x(1)e^{-j 2\pi f} + x(2)e^{-j 2\pi \cdot 2f} + \cdots + x(N - 1) e^{-j 2\pi \cdot (N - 1)\,f}
$$
The period is decided by $x(1)e^{-j 2\pi f}$ and is $1$ here. We should sample it in $[0, 1)$. Assuming $f = \frac{k}{N} \quad k \in [0, N)$, we have
\begin{equation*}
\widetilde{x}_s(\frac{k}{N}) = \widetilde{x}_s(k) = \sum_{n = 0}^{N - 1} \; x(n)  \, e^{-j 2\pi n \cdot \frac{k}{N}}
\label{eq:DFT}
\end{equation*}
$x_s(k)$ is also discrete now, so that it can be processed with the computer.


\section{Analog Frequency in DFT}
Variables in (\ref{eq:DFT}) are $n$ and $k$.

First, here's my thoughts about $n$. Since $\widetilde{x}_s(k)$ is just a sampling of $\widetilde{x}_s(f)$, if we can find high analog frequency in $\widetilde{x}_s(f)$, the same method works for $\widetilde{x}_s(k)$. Go back and see (\ref{eq:relationship}), it tells us that $\widetilde{x}_s(f)$ is just many copies and frequency-shift of $\widetilde{x}_a(f)$. Therefore, the bigger $n$ gets, the higher $\widetilde{x}_s(f)$ reach. 

Then, $k$ also matters. $k$ has its source from $f$, when $k$ (also $f$) gets bigger, no doubt the frequency gets bigger. However, $n$ decides on frequency domain which period we are in. If $n$ is fixed, we cannot reach other other frequency out of this specific interval.  


\section{Conclusion}



\bibliographystyle{ieeetr}
\bibliography{../bib/database}

\begin{appendices}

\end{appendices}

\end{document}