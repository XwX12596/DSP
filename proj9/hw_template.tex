\documentclass{article}
\usepackage{amsmath, amssymb, cite, algorithmic, url, braket}
\usepackage{graphicx}
\usepackage{pythonhighlight}
\usepackage[margin=1.5cm]{geometry}
\usepackage[title]{appendix}
\usepackage{subfigure}
\usepackage{listings}
\usepackage{booktabs}

\graphicspath{{../pic/}}
\lstset{
language=[ANSI]{C},
showtabs=true,
tab=,
tabsize=2,
basicstyle=\ttfamily\footnotesize,%\setstretch{.5},
stringstyle=\color{stringcolour},
showstringspaces=false,
alsoletter={1234567890},
otherkeywords={\%, \}, \{, \&, \|},
keywordstyle=\color{keywordcolour}\bfseries,
upquote=true,
morecomment=[s]{/*}{*/},
commentstyle=\color{commentcolour}\slshape,
literate=*%
{=}{{\literatecolour=}}{1}%
{-}{{\literatecolour-}}{1}%
{+}{{\literatecolour+}}{1}%
{*}{{\literatecolour*}}{1}%
{!}{{\literatecolour!}}{1}%
{[}{{\literatecolour[}}{1}%
{]}{{\literatecolour]}}{1}%
{<}{{\literatecolour<}}{1}%
{>}{{\literatecolour>}}{1}%
% {>>>}{\pythonprompt}{3}%
,%
frame=trbl,
rulecolor=\color{black!40},
backgroundcolor=\color{white},
breakindent=.5\textwidth,frame=single,breaklines=true
}

\begin{document}
\title{DSP Homework 09}
\author{Xu, Minhuan}
\maketitle
\tableofcontents
\begin{abstract}

\end{abstract}

\section{1}

\section{Digital Number Representations}

\subsection{Exact Meanings of Common Representations}
\begin{enumerate}
	\item[-] byte: 
	\item[-] short integer: 
	\item[-] integer: 
	\item[-] float: 
	\item[-] double: 
	\item[-] quadruple types: 
	\item[-] fixed-point: 
	\item[-] floating-point: 
\end{enumerate}

\section{3}

\section{Optimal Quantization Strategy When PDF Has Uniform Distribution}
We have the quantization error of J which can be described as below:
\begin{equation}
J = \sum_{i = 1}^{M}  \int_{b_{i - 1}}^{b_i} [Qi(x) - x]^2 \, p(x) \; \mathrm{d}x
\end{equation}

In the special case of uniform distribution, we Have
\begin{equation}
J = \sum_{i = 1}^{M}  \int_{b_{i - 1}}^{b_i} [q_i - x]^2 \; \mathrm{d}x
\label{eq:label}
\end{equation}

To find the best $q_i$, take the partial derivative of $q_i$
\begin{equation*}
	\begin{aligned}
		\frac{\partial J}{\partial q_i} &= \sum_{i = 1}^{M} \int_{b_{i - 1}}^{b_i} \frac{\partial}{\partial q_i} (q_i^2 - 2q_ix + x^2) \; \mathrm{d}x \\
		&= \sum_{i = 1}^{M} \int_{b_{i - 1}}^{b_i} (2 q_i - 2x) \; \mathrm{d}x \\ 
	\end{aligned}
\end{equation*}

We want $\frac{\partial J}{\partial q_i} = 0$. Therefore \\ 
\begin{equation*}
	\begin{aligned}
		 q_i \int_{b_{i - 1}}^{b_i} \; \mathrm{d}x &=  \int_{b_{i - 1}}^{b_i} x \; \mathrm{d}x \\ 
		 q_i (b_i - b_{i - 1}) &= \frac12 \;  (b_i^2 - b_{i - 1}^2) 	 
	\end{aligned}
\end{equation*}

Therefore
\begin{equation}
	q_i = \frac{b_i + b_{i - 1}}{2} 
	\label{eq:qiResult}
\end{equation}

And, the same as the $q_i$, take the partial of $b_i$, and make $\frac{\partial J}{\partial b_i} = 0$
\begin{equation*}
	\begin{aligned}
		\frac{\partial J}{\partial b_i} &= \frac{\partial}{\partial b_i} \sum_{i = 1}^{M} \int_{b_{i - 1}}^{b_i} (q_i - x)^2 \; \mathrm{d}x \\ 
		&= \frac{\partial}{\partial b_i} [\int_{b_{i - 1}}^{b_i} (q_i - x)^2 \; \mathrm{d}x + \int_{b_{i}}^{b_{i + 1}} (q_{i + 1} - x)^2 \; \mathrm{d}x] \\ 
		&= 0
	\end{aligned}
	\label{eq:biCondi}
\end{equation*}

We learned (\ref{eq:derivation}) in out freshman year that
\begin{equation}
\begin{aligned}
	\frac{d}{dx} \int_{a}^{x} \, f(t) \; \mathrm{d}t &= f(x) \\ 
	\frac{d}{dx} \int_{x}^{a} \, f(t) \; \mathrm{d}t &= - f(x)
\end{aligned}
\label{eq:derivation}
\end{equation}

Rewrite the (\ref{eq:biCondi}) 
\begin{equation*}
	\begin{aligned}
		(q_i - b_i)^2 &= (q_{i + 1} - b_i)^2 \\ 
		b_i - q_i &= q_{i + 1} - b_i
	\end{aligned}
\end{equation*}

We can have the other result
\begin{equation}
	b_i = \frac{q_{i + 1} + q_i}{2}
	\label{eq:biResult}
\end{equation}

If we combine (\ref{eq:qiResult}) with (\ref{eq:biResult}). First, we can know that 
\begin{gather}
	q_i = \frac{b_i + b_{i - 1}}{2} \notag \\ 
	q_{i + 1} = \frac{b_{i + 1} + b_i}{2} \notag
\end{gather}

Put them in (\ref{eq:biResult})
\begin{equation*}
	\begin{aligned}
		b_i &= \frac{q_i + q_{i + 1}}{2} \\ 
		&= \frac{1}{2} \; (b_i + \frac{b_{i - 1} + b_{i + 1}}{2}) \\ 
	\end{aligned}
\end{equation*}

Therefore
$$
	b_i = \frac{b_{i - 1} + b_{i + 1}}{2}
$$

We can easily know that $b_i$ is an arithmetic sequence, and because (\ref{eq:qiResult}), $q_i$ is an arithmetic sequence too. 

In conclusion, if $p(x)$ is in the special case of uniform distribution, the range of $[0, 1]$ should be equally divided into M parts, and $q_i$ should be the mean of $b_i$ and $b_{i + 1}$.

\section{Conclusion}



\bibliographystyle{ieeetr}
\bibliography{../bib/database}

\begin{appendices}

\end{appendices}

\end{document}