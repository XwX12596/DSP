\documentclass{article}
\usepackage{amsmath, amssymb, cite, algorithmic, url, braket}
\usepackage{graphicx}
\usepackage{pythonhighlight}
\usepackage[margin=1.5cm]{geometry}
\usepackage[title]{appendix}
\usepackage{subfigure}
\usepackage{listings}
\usepackage{booktabs}

\graphicspath{{../pic/}}
\lstset{
language=[ANSI]{C},
showtabs=true,
tab=,
tabsize=2,
basicstyle=\ttfamily\footnotesize,%\setstretch{.5},
stringstyle=\color{stringcolour},
showstringspaces=false,
alsoletter={1234567890},
otherkeywords={\%, \}, \{, \&, \|},
keywordstyle=\color{keywordcolour}\bfseries,
upquote=true,
morecomment=[s]{/*}{*/},
commentstyle=\color{commentcolour}\slshape,
literate=*%
{=}{{\literatecolour=}}{1}%
{-}{{\literatecolour-}}{1}%
{+}{{\literatecolour+}}{1}%
{*}{{\literatecolour*}}{1}%
{!}{{\literatecolour!}}{1}%
{[}{{\literatecolour[}}{1}%
{]}{{\literatecolour]}}{1}%
{<}{{\literatecolour<}}{1}%
{>}{{\literatecolour>}}{1}%
% {>>>}{\pythonprompt}{3}%
,%
frame=trbl,
rulecolor=\color{black!40},
backgroundcolor=\color{white},
breakindent=.5\textwidth,frame=single,breaklines=true
}

\begin{document}
\title{DSP Homework}
\author{Xu, Minhuan}
\maketitle
\tableofcontents
\begin{abstract}

\end{abstract}

\section{01}

\section{IDFT}
To derive the IDFT formula, we only should derive the expression of $A^{-1}$. And since $A$ can be expressed as below
$$
\begin{bmatrix}
 1 & 1 & 1 & \cdots & 1 \\
 1 & w^1 & w^2 & \cdots & w^{N - 1} \\
 1 & w^2 & w^4 & \cdots & w^{2(N - 1)}\\ 
 \vdots & \vdots & \vdots& \ddots & \vdots\\
 1 & w^{2(N - 1)} & w^{4(N - 1)} & \cdots & w^{(N - 1)^2}
 \end{bmatrix}
$$
we note rows in A as $\alpha_i$, and rewrite it as below
$$
A = 
\begin{bmatrix}
\alpha_0 \\ \alpha_1 \\ \alpha_2 \\ \vdots \\ \alpha_{N - 1}
 \end{bmatrix} 
$$
See $\alpha_i \cdot \alpha_j^H$, if $i \neq j$, we let $\lambda = i - j \in (0, N)$, and
\begin{equation}
\begin{aligned}
    \alpha_i \cdot \alpha_j^H &= \begin{bmatrix}
    1 & w^i & w^{2i} & w^{3i} & \cdots & w^{(N - 1)i}
     \end{bmatrix}
     \cdot
     \begin{bmatrix}
    1 \\ w^j \\ w^{2j} \\ w^{3j} \\ \vdots \\ w^{(N - 1)j}
     \end{bmatrix}\\
     &= 1 + w^{\lambda} + w^{2 \lambda} + \cdots + w^{(N - 1) \lambda}\\
     &= \frac{1 - w^{N \lambda}}{1 - w^{\lambda}}\\ 
     &= \frac{1 - e^{j 2\pi\lambda} \quad (= 0)}{1 - e^{-j\frac{2\pi}{N}\lambda} \quad (\neq 0)}\\ 
     &= 0
\end{aligned}
\label{eq:inj}
\end{equation}
If $i = j$, we have
\begin{equation}
    \alpha_i \cdot \alpha_j^H = 1 + w^{0} + w^{2 \times 0} + \cdots + w^{(N - 1) \times 0} = N
    \label{eq:ij}
\end{equation}
So, conclude (\ref{eq:inj}) and (\ref{eq:ij}), we have
\begin{equation}
    \alpha_i \cdot \alpha_j^H =
    \begin{cases}
        0 & \text{if } i \neq j \\ 
        N & \text{if } i = j
    \end{cases}
\label{eq:multi_ij}
\end{equation}
Therefore, see $A \cdot A^H$
\begin{equation}
\begin{aligned}
    A \cdot A^H &= 
    \begin{bmatrix}
        \alpha_0 \\ \alpha_1 \\ \alpha_2 \\ \vdots \\ \alpha_{N - 1}
    \end{bmatrix}
    \cdot
    \begin{bmatrix}
        \alpha_0^H & \alpha_1^H & \alpha_2^H & \cdots & \alpha_{N - 1}^H
    \end{bmatrix} \\ 
    &= N \cdot 
    \begin{bmatrix}
        \alpha_0 \cdot \alpha_0^H & \alpha_0 \cdot \alpha_1^H & \cdots & \alpha_0 \cdot \alpha_{N - 1}^H \\ 
        \alpha_1 \cdot \alpha_0^H & \alpha_1 \cdot \alpha_1^H & \cdots & \alpha_1 \cdot \alpha_{N - 1}^H \\ 
        \vdots & \vdots & \ddots & \vdots \\ 
        \alpha_{N - 1} \cdot \alpha_0^H & \alpha_{N - 1} \cdot \alpha_1^H & \cdots & \alpha_{N - 1} \cdot \alpha_{N - 1}^H
    \end{bmatrix} \\
    &= N \cdot 
    \begin{bmatrix}
        1 & 0 & \cdots & 0 \\ 
        0 & 1 & \cdots & 0 \\ 
        \vdots & \vdots & \ddots & \vdots \\ 
        0 & 0 & \cdots & 1
    \end{bmatrix} \\
    &= N \cdot I
\end{aligned}
\label{eq:AAH}
\end{equation}
If we multiply $A^{-1}$ in both side of (\ref{eq:AAH}), we have
\begin{equation}
    A^{-1} = \frac1N A^H
\label{eq:A-1}
\end{equation}
Therefore, IDFT formula can be expressed as below
\begin{equation}
    \begin{aligned}
        x &= A^{-1} \tilde{x}\\ 
        &= \frac1N A^H \tilde{x} \\ 
        &= \frac{1}{N} (A \tilde{x}^H)^H \\ 
        &= \frac{1}{N} \sum^{N - 1}_{k = 0} \tilde{x}(k)e^{j2\pi kn /N}
    \end{aligned}
\end{equation}
Also, a more symmetric definition is as below
$$
    \tilde{x}(k) = \frac{1}{\sqrt{N}} \sum_{n = 0}^{N - 1} x(n) e^{-j 2\pi kn/N}
$$
$$
    x(n) = \frac{1}{\sqrt{N}} \sum_{k = 0}^{N - 1} \tilde{x}(k) e^{j 2\pi kn/N}
$$
\section{Find Analog Spectrum Value in DFT}
The DTFT formula can be rewritten as below
\begin{equation}
\begin{aligned}
\tilde{x}_s(f) &= \mathcal{F} \left[ x_s(t) \right] \\ 
&= \mathcal{F} \left[ x_a(t) \times s(t) \right] \\ 
&= \tilde{x}_a(f) *\tilde{s}(f) \\ 
&= \tilde{x}_a(f) * \left[ f_s \sum_n \delta(f - nf_s) \right] \\
&=f_s \cdot \sum_n \tilde{x}_a(f - nf_s) \\
\end{aligned}
\label{eq:dtft_rewrite}
\end{equation}
To make the frequency discrete, we should do sampling in frequency domain. The period of (\ref{eq:dtft_rewrite}) is $f_s$, so we make the analog frequency be $f = f_s \cdot \frac k N$.

Therefore, we have
\begin{equation}
\begin{aligned}
\tilde{x}(\frac{k\cdot f_s}{N}) &= f_s \cdot \sum_n \tilde{x}_a[f_s \cdot \frac kN - f_s \cdot n] \\ 
&= f_s \cdot \sum_n \tilde{x}_a[f_s(\frac kN - n)]
\end{aligned}
\label{eq:dtft_rewrite_2}
\end{equation}
In order to compute the exact value of $\tilde{x}_a(f)$ for $f = 3010.3 ~\mathrm{Hz} $, we should make $f_s(\frac kN - n) = 3010.3 ~\mathrm{Hz}$ in (\ref{eq:dtft_rewrite_2}). Usually, we know that $f_s > B$ where $B$ is the bandwidth of $x_a(t)$, so $n$ must be $0$, therefore, in order to compute the exact frequency spectrum value, we only should to compute: 
$$
f_s\cdot \frac kn = 3010.3 ~\mathrm{Hz}
$$


\section{Comparison of Codes Running Speed}
Assuming $N$ is the total number of the sampling points, I drew a picture of how much time it takes for 3 methods to do the calculations. The x-axis is $\mathrm{Log}N$, and the y-axis is Time $t$. See Fig.~\ref{fig:speedtest}

\begin{figure}[!h]
    \centering
    \includegraphics[width=5 in]{../pic/speedtest.png}
    \caption{Comparison of the Codes Running Speed}
    \label{fig:speedtest}
\end{figure}


\bibliographystyle{ieeetr}
\bibliography{../bib/database}

\begin{appendices}
\section{Code Listing}
\begin{python}
import numpy as np
import time
from matplotlib import pyplot as plt

def dft(x):
    N = len(x)
    xt = np.matmul(
        np.exp(-1j * 2 * np.pi * np.matmul(np.arange(N).reshape(N, 1), np.arange(N).reshape(1, N)/N)),
        x
        )
    return xt

def idft(xt):
    return np.conj(dft(np.conj(xt))) / N

def fft(x):
    N = len(x)
    if N == 1:
        return x
    else:
        x0, x1 = x[::2] , x[1::2]
        xt0 = fft(x0)
        xt1 = fft(x1)
        tmp = np.exp(-1j*2*np.pi*np.arange(N/2)/N) * xt1
        xt = np.concatenate([xt0 + tmp, xt0 - tmp])
    return xt

def ifft(xt):
    return np.conj(fft(np.conj(xt))) / N

def test(m):
    x = np.random.rand(2**m)

    # MyDFT
    st1 = time.time()
    y1 = dft(x)
    et1 = time.time()
    # xx1 = idft(y1)

    # MyFFT
    st2 = time.time()
    y2 = fft(x)
    et2 = time.time()
    # xx2 = idft(y2)

    # LibFFT
    st3 = time.time()
    y3 = np.fft.fft(x)
    et3 = time.time()
    # xx3 = idft(y3)

    # err1 = xx1 - x
    # err2 = xx2 - x
    # err3 = xx3 - x

    t1 = et1 - st1
    t2 = et2 - st2
    t3 = et3 - st3
    
    return t1, t2, t3


M = [i for i in range(6, 13)]
T1 = []
T2 = []
T3 = []
for m in M:
    tmp1, tmp2, tmp3 = test(m)
    T1.append(tmp1)
    T2.append(tmp2)
    T3.append(tmp3)

fig = plt.figure()
plt.xlabel('LogN')
plt.ylabel('Time')
plt.subplot(211)
plt.scatter(M, T1, label='MyDFT', c='r')
plt.scatter(M, T2, label='MyFFT', c='g')
plt.subplot(212)
plt.scatter(M, T2, c='g')
plt.scatter(M, T3, label='LibFFT', c='b')
fig.legend()
plt.show()
\end{python}
\end{appendices}

\end{document}